\documentclass{beamer}
\mode<presentation>
\usepackage{amsmath,amssymb}
\usepackage{color}
\usepackage{etoolbox}
\usepackage[dutch]{babel}
\usepackage{graphicx}
\usepackage{mathtools}
\usetheme{Rochester}
\usecolortheme{beaver}
\setbeamertemplate{footline}{\insertframenumber/\inserttotalframenumber}

%\setbeamercolor{normal text}{bg=black,fg=white}
%\setbeamercolor{frametitle}{bg=black,fg=white}
%\setbeamercolor{alerted text}{bg=black,fg=white}
%\setbeamercolor{title}{bg=black,fg=white}
%\setbeamercolor{stelling}{bg=black,fg=white}

\title{Adaptieve benaderingsmethodes}
\author{Jan Westerdiep}

\newcommand{\R}{\mathbb{R}}
\newcommand{\N}{\mathbb{N}}
\newcommand{\Z}{\mathbb{Z}}
\newcommand{\C}{\mathbb{C}}
\newcommand{\A}{\mathbb{A}}
\newcommand{\Q}{\mathbb{Q}}
\newcommand{\F}{\mathbb{F}}
\newcommand{\f}{\varphi}
\newcommand{\e}{\varepsilon}
\renewcommand{\d}{\delta}

\newtheorem{bewijs}{Bewijs}
\newtheorem{voorbeeld}{Voorbeeld}
\newtheorem{stelling}{Stelling}
\newtheorem{definitie}{Definitie}
\newtheorem{gevolg}{Gevolg}
\newtheorem{opmerking}{Opmerking}

\setbeamerfont{block title}{size=\small}

\begin{document}

\begin{frame}
\titlepage
\end{frame}

\begin{frame}{Wat?}
\begin{itemize}
  \item \alert<3>{Adaptieve} \alert<2>{benaderingsmethodes} \pause
  \item Functies benaderen \pause
  \item Door te kijken \emph{waar} de functie moeilijk is en daar in te zoomen \pause
  \item Dit alles aan de hand van twee artikelen van Peter Binev \pause
\end{itemize}
\end{frame}

\begin{frame}{2004 -- Iets preciezer}
\begin{columns}[t]
\begin{column}[T]{\linewidth - 2.5cm}
\begin{itemize}
  \item Pak $f: [a,b] \to \R$ met een opdeling van het domein en laat $n \in \N$ vast
  \uncover<2->{\item Op elk stukje, best benaderende polynoom van graad $n$}
  \uncover<3->{\item \emph{Kies} stukje om op te delen}
  \uncover<4->{\item Herhaal}
  \uncover<5->{\item[$\Rightarrow$] Tree Generating Algorithm}
\end{itemize}
\end{column}
\begin{column}[T]{2.5cm}
  \centering
  \includegraphics[width=\linewidth]{schets_0.png}\\
  \uncover<2->{\includegraphics[width=\linewidth]{schets_1.png}\\}
  \uncover<3->{\includegraphics[width=\linewidth]{schets_2.png}\\}
  \uncover<4->{\includegraphics[width=\linewidth]{schets_3.png}\\}
\end{column}
\end{columns}
\centering
  \uncover<5->{\includegraphics[width=0.8\linewidth]{lol.pdf}}
\end{frame}

\begin{frame}{Enkel door midden hakken}
  \centering
  \only<1>{\includegraphics[height=0.5\textheight]{41.png}\\
  \includegraphics[height=0.5\textheight]{tree.pdf}}
  \only<2>{\includegraphics[height=0.5\textheight]{42.png}\\
  \includegraphics[height=0.5\textheight]{tree_1.pdf}}
  \only<3>{\includegraphics[height=0.5\textheight]{43.png}\\
  \includegraphics[height=0.5\textheight]{tree_2.pdf}}
  \only<4>{\includegraphics[height=0.5\textheight]{44.png}\\
  \includegraphics[height=0.5\textheight]{tree_3.pdf}}
  \only<5>{\includegraphics[height=0.5\textheight]{45.png}\\
  \includegraphics[height=0.5\textheight]{tree_4.pdf}}
  \only<6>{\includegraphics[height=0.5\textheight]{46.png}\\
  \includegraphics[height=0.5\textheight]{tree_5.pdf}}
  \only<7>{\includegraphics[height=0.5\textheight]{47.png}\\
  \includegraphics[height=0.5\textheight]{tree_6.pdf}}
  \only<8>{\includegraphics[height=0.5\textheight]{48.png}\\
  \includegraphics[height=0.5\textheight]{tree_7.pdf}}
\end{frame}

\begin{frame}{2004 -- Iets preciezer}
  \begin{itemize}
    \item Bewijsbaar: Binev-2004 maakt een boom die na $n$ opdelingen
      \begin{itemize}
        \item gevonden is in $\mathcal{O}(n)$, en
        \item beter is dan de optimale boom in $n/2$ opdelingen
      \end{itemize}
  \end{itemize}
\end{frame}

\begin{frame}{2013 -- Iets preciezer}
\begin{columns}[t]
\begin{column}[T]{\linewidth - 2.5cm}
\begin{itemize}
  \item Pak $f: [a,b] \to \R$ met een opdeling van het domein
  \uncover<2->{\item Op elk stukje, best benaderende polynoom van bepaalde graad}
  \uncover<3->{\item \emph{Kies} stukje om op te delen}
  \uncover<5->{\item[$\Rightarrow$] \emph{Of} verhoog de graad van het polynoom}
  \uncover<4->{\item Herhaal}
\end{itemize}
\end{column}
\begin{column}[T]{2.5cm}
  \centering
  \includegraphics[width=\linewidth]{schets_0.png}\\
  \uncover<2->{\includegraphics[width=\linewidth]{schets_1.png}\\}
  \uncover<3->{\includegraphics[width=\linewidth]{schets_2.png}\\}
  \uncover<4->{\includegraphics[width=\linewidth]{schets_3.png}\\}
\end{column}
\end{columns}
\end{frame}

\begin{frame}{Ook polynomiale graad verhogen}
  \centering
  \only<1>{\includegraphics[height=0.5\textheight]{31.png}\\
  \includegraphics[height=0.5\textheight]{tree_hp_1.pdf}}
  \only<2>{\includegraphics[height=0.5\textheight]{32.png}\\
  \includegraphics[height=0.5\textheight]{tree_hp_2.pdf}}
  \only<3>{\includegraphics[height=0.5\textheight]{33.png}\\
  \includegraphics[height=0.5\textheight]{tree_hp_3.pdf}}
  \only<4>{\includegraphics[height=0.5\textheight]{34.png}\\
  \includegraphics[height=0.5\textheight]{tree_hp_4.pdf}}
  \only<5>{\includegraphics[height=0.5\textheight]{35.png}\\
  \includegraphics[height=0.5\textheight]{tree_hp_5.pdf}}
  \only<6>{\includegraphics[height=0.5\textheight]{36.png}\\
  \includegraphics[height=0.5\textheight]{tree_hp_6.pdf}}
  \only<7>{\includegraphics[height=0.5\textheight]{37.png}\\
  \includegraphics[height=0.5\textheight]{tree_hp_7.pdf}}
  \only<8>{\includegraphics[height=0.5\textheight]{38.png}\\
  \includegraphics[height=0.5\textheight]{tree_hp_8.pdf}}
\end{frame}

\begin{frame}{Ander voorbeeld}
  \centering
  \only<1>{\includegraphics[width=0.5\linewidth]{11.png}
  \includegraphics[width=0.5\linewidth]{21.png}}
  \only<2>{\includegraphics[width=0.5\linewidth]{12.png}
  \includegraphics[width=0.5\linewidth]{22.png}}
  \only<3>{\vspace{-1em}\includegraphics[width=0.5\linewidth]{14.png}
  \includegraphics[width=0.5\linewidth]{24.png}}
  \only<4>{\vspace{-1em}\includegraphics[width=0.5\linewidth]{18.png}
  \includegraphics[width=0.5\linewidth]{28.png}}
\end{frame}

\begin{frame}{Bij wie?}
  \begin{columns}[t]
    \begin{column}[T]{\linewidth - 6cm}
      Rob Stevenson
      \begin{itemize}
        \item Wiskunde -- Theorie
      \end{itemize}
      ~~\\
      \uncover<2>{
        Dick van Albada
        \begin{itemize}
          \item Informatica -- Praktijk
        \end{itemize}
      }
    \end{column}
    \begin{column}[T]{3cm}
      \uncover<2>{
        \includegraphics[width=2.5cm]{Albada.jpg}
      }
    \end{column}
    \begin{column}[T]{3cm}
      \includegraphics[width=2.5cm]{Stevenson.jpg}
    \end{column}
  \end{columns}
\end{frame}

\begin{frame}{Waarom?}
  \begin{itemize}
    \item Wiskunde op de computer: goede samenkomst wiskunde/informatica \pause
    \item Spiksplinternieuw (2013) \pause
    \item Vraagstelling is makkelijk, de theorie erachter niet \pause
    \item Veel uitbreiding en toepassing mogelijk
  \end{itemize}
\end{frame}

\begin{frame}{Wanneer?}
  \begin{itemize}
    \item Februari, maart, april: Theorie begrijpen, implementatie in Python \pause
    \item Mei: Implementatie in C \pause
    \item Daarna: High Performance, parallelliseren, toepassen
  \end{itemize}
\end{frame}

\begin{frame}{Jan Westerdiep -- Adaptieve benaderingsmethodes}
  \centering
  \includegraphics[width=0.4\linewidth]{42.png}
  \includegraphics[width=0.4\linewidth]{48.png}\\
  \includegraphics[width=0.4\linewidth]{32.png}
  \includegraphics[width=0.4\linewidth]{38.png}\\
\end{frame}

\begin{frame}{Schets van inhoudsopgave}
  \begin{enumerate}
    \item 1-dimensionale geval \pause
    \item 2-dimensionale geval \pause
    \item Implementatie in C \pause
    \item Parallelliseren \pause
    \item Implementatie in e.o.a.~parallelle situatie (GPU, supercomputer) \pause
    \item (wellicht) Uitbreiding en toepassing
  \end{enumerate}
\end{frame}

\end{document}
